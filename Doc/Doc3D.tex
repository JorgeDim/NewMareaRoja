%&latex
\documentclass[12pt]{article}
\usepackage[T1]{fontenc}
%\usepackage{unicode-math}
%\setmathfont{XITS Math}
\usepackage{amsfonts,amsmath}
\usepackage{geometry}
\usepackage{color}
\usepackage{hyperref}
\geometry{letterpaper,lmargin=0.5cm,rmargin=3cm,tmargin=2cm,bmargin=1cm}
\def\N{\mathbb N}
\def\Z{\mathbb Z}
\def\I{\mathbb I}
\def\Q{\mathbb Q}
\def\R{\mathbb R}
\def\di{\displaystyle}

\catcode`@=11
\def\sen{\mathop {\operator@font sen}\nolimits}
\def\div{\mathop {\operator@font div}\nolimits}
\usepackage{graphicx}

\begin{document}

 


\section{Ecuaci�n de difusi�n-convecci�n}
Ecuaci�n a tratar:

\begin{equation}
a(x,t)C(x,t)+\div (C \vec V)-{\rm div}(  D\nabla C)=F.
\end{equation}
Integrando sobre un volumen $\omega$ se obtiene que:
\begin{eqnarray}\label{Vint}
\int_\omega aCd\omega+\int_{\partial \omega}C (\vec V
\cdot \vec n)\, d\Sigma-\int_{\partial \omega } D(\nabla C \cdot\vec n) \,d\Sigma &=& \int_\omega Fd\omega.  
\end{eqnarray}

Forma discreta: Para todo $\omega$ tal que $\partial\omega=\bigcup\limits_{j=1}^M\Sigma_j$ tenemos que:
\begin{eqnarray*}
\langle aC\rangle_\omega|\omega|+\sum_j C _{\Sigma_j}(\vec V_{\Sigma_j}
\cdot \vec n_j)\,
|\Sigma_j| -\sum_j D \frac {C_{\omega_j}-C_\omega}{d_{\omega \to\omega_ j}} \,|\Sigma_j| &=& \langle F\rangle_\omega|\omega|
\end{eqnarray*}
Usando la condici�n UPWIND:
$$
C_{\Sigma j}=\begin{cases} 
C_\omega &\mbox{si }(\vec V_{\Sigma_j}
\cdot \vec n_j)\geq 0
\\
C_{\omega_j} &\mbox{si }(\vec V_{\Sigma_j}
\cdot \vec n_j)< 0.
\end{cases}
$$
la ecuaci�n queda
\begin{eqnarray*}
\langle aC\rangle_\omega|\omega|+\sum_{j:\atop \vec V_{\Sigma_j}
\cdot \vec n_j\geq 0}C _{\omega}(\vec V_{\Sigma_j}
\cdot \vec n_j)\,
|\Sigma_j| +\sum_{j:\atop \vec V_{\Sigma_j}
\cdot \vec n_j< 0}C _{\omega_j}(\vec V_{\Sigma_j}
\cdot \vec n_j)\,
|\Sigma_j| -\sum_j D \frac {C_{\omega_j}-C_\omega}{d_{\omega \to\omega_ j}} \,|\Sigma_j| &=& \langle F\rangle_\omega|\omega|
\end{eqnarray*}
de donde se despeja:
\begin{eqnarray*}
C_\omega&=&
\frac{\di\sum_{j:\atop \vec V_{\Sigma_j}
\cdot \vec n_j< 0}(-\vec V_{\Sigma_j}
\cdot \vec n_j)\,
|\Sigma_j|C_{\omega_j} +\sum_j  \frac {D}{d_{\omega \to\omega_ j}} \,|\Sigma_j|C_{\omega_j}+\langle F\rangle_\omega|\omega|}
{\di\sum_{j:\atop \vec V_{\Sigma_j}
\cdot \vec n_j\geq 0}(\vec V_{\Sigma_j}
\cdot \vec n_j)\,
|\Sigma_j| +\sum_j  \frac {D}{d_{\omega \to\omega_ j}} \,|\Sigma_j|+\langle a\rangle_\omega|\omega| \,} 
\end{eqnarray*}
En esta ecuaci�n, (dado que en general el campo de velocidades no tiene divergencia nula) es conveniente usar: $\vec V_{\Sigma_j}=\frac 12(\vec V_\omega+\vec V_{\omega_j})$

\subsection{Condici�n de borde en pared Convectiva:}

Si $\partial V=\Sigma_i\cup\Sigma_c$ est� formada por una zona interna ($\Sigma_i$) y una zona de borde convectiva $(\Sigma_c$), entonces la ecuaci�n \eqref{Vint} se transforma en:
\begin{eqnarray}\label{Vc}
0&=&\int_{\partial V}(\rho c)_f \frac{\partial \psi}{\partial n}
(T-T_\omega)\, d\Sigma+\int_{\Sigma_i } k_m\nabla T .\vec n \,d\Sigma-\int_{\Sigma_c } \vec q .\vec n \,d\Sigma.  
\end{eqnarray}
Donde $\vec q=\tilde h(T-T_{ext})\vec n$. Reemplazando en \eqref{Vc} queda:
\begin{eqnarray}\label{Vc}
0&=&\int_{\partial V}(\rho c)_f \frac{\partial \psi}{\partial n}
(T-T_\omega)\, d\Sigma+\int_{\Sigma_i } k_m\nabla T .\vec n \,d\Sigma+\int_{\Sigma_c } \tilde h(T_{ext}-T)  \,d\Sigma.  
\end{eqnarray}
La forma discreta (usando UPWIND) es ahora: Para todo $V$ 
\begin{eqnarray*}
0&=&\sum_{j:\atop \psi_j>\psi_\omega} (\rho c)_f \frac {\psi_j-\psi_\omega}{d_{\omega \to j}} (T_{j}-T_\omega)\,
|\Sigma_j| +\sum_{j:\atop\Sigma_j\subseteq\Sigma_i} k_m \frac {T_j-T_\omega}{d_{\omega \to j}} \,|\Sigma_j|+\sum_{j:\atop\Sigma_j\subseteq\Sigma_c} \tilde h(T_{ext}-T_\omega) \,|\Sigma_j|
\end{eqnarray*}
de donde, despejando $T_\omega$ queda:
\begin{eqnarray*}
T_\omega&=&
\frac{\di\sum_{j:\atop \psi_j>\psi_\omega}(\rho c)_f \frac {\psi_j-\psi_\omega}{d_{\omega \to j}} \,
|\Sigma_j| \cdot T_{ j}\, + \sum_{j:\atop\Sigma_j\subseteq\Sigma_i}  \frac {k_m|\Sigma_j|}{d_{\omega \to j}}\cdot T_{ j}+\sum_{j:\atop\Sigma_j\subseteq\Sigma_c} \tilde h|\Sigma_j|\cdot T_{ext}}
{\di\sum_{j:\atop \psi_j>\psi_\omega} (\rho c)_f \frac {\psi_j-\psi_\omega}{d_{\omega \to j}} \,
|\Sigma_j|+ \sum_{j:\atop\Sigma_j\subseteq\Sigma_i}  \frac {k_m|\Sigma_j|}{d_{\omega \to j}} +\sum_{j:\atop\Sigma_j\subseteq\Sigma_c} \tilde h|\Sigma_j|} 
\end{eqnarray*}
\end{document}


